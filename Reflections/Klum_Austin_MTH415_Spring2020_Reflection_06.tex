\documentclass[10pt]{article}


\usepackage{amsmath, amssymb, amsthm, amscd, amsfonts, dsfont}
\usepackage{mathrsfs}
\usepackage{graphicx,array}
\usepackage{verbatim}
\usepackage{enumerate}
\usepackage{url,hyperref}
\usepackage{comment}
\usepackage{multicol}
\usepackage{tikz}
\usepackage{charter}
\usepackage{bbold}

\setlength\extrarowheight{6pt}


\pagestyle{empty}
\thispagestyle{empty}
%%%%%%%%%%%%%%%%%%%%%%%%%%%%%%%%%%%%%%%%%%%%%%%%%%%%%%%%%%%%%%%%%%%%%
%                         ADJUST MARGINS
\usepackage[margin=0.67in]{geometry} % Makes margins 1in. Remove or comment out (with percent) if you want.

\begin{document}

\begin{center}
\Large \bfseries MTH 415 \textcolor{blue}{UPDATED} Weekly Reflection Activity Cover Sheet\\ 
{\it Due \underline{every Monday} by 11:30pm in Canvas}\\ 
\smallskip
\it Name/Section/Date/Submission \# \fbox{{\Large Austin Klum MTH-415 2020/03/23 }}
\end{center}

{\bf Effective learners} regularly self-assess their progress and then adjust study strategies in response. This reflection will help you to analyze your week's performance and help you on your path to improving your learning strategies.

\begin{enumerate}

\item What study techniques have you used in the last week:\\
	
	\begin{tabular}{|c|c|c|c|}
		\hline
		 {\bf Strategy} & {\bf Never} & {\bf Once} & {\bf Multiple (How many?)}\\
		\hline
		Identified and Removed Distractions While Studying &&X&\\
		\hline
		Studied in well-spaced blocks of time &X&&\\
		\hline
		Studied in blocks of time not exceeding 60 mins &X&&\\
		\hline
		Worked on Making/Taking Self-Timed Short Quizzes &X&&\\
		\hline
		\hline


		Studied Theorems and Definitions in Text&&X&\\
		\hline 
		Made Notes While Studying Text \& Online&&X&\\
		\hline 
		Worked on Error Analysis&&X&\\
		\hline 
		\hline
		 
		Studied and Reworked Exemplary Worked Examples in Text &X&&\\
		\hline 
		Studied and Reworked Examples/Proofs from Lecture Notes &X&&\\
		\hline 
		Studied and Reworked Worked Examples found Online &X&&\\
		\hline 
		\hline


		Made/Generated/Produced Questions this week &&X&\\
		\hline
		Answered Questions generated this week &X&&\\
		\hline
		\hline
		
		Online Office Hours with Dr. Das &X&&\\
		\hline		
		Asked Questions during Online Office Hours &X&&\\
		\hline
		\hline
		
		Other strategies (may describe in reflection) &&X&\\
		&&&\\
		\hline
		
			
	\end{tabular}

\item How many hours did you schedule for MTH 415 study this week? How many hours did you actually study this week? Suggestions for improvement, if appropriate.
\vfill
I didn't really schedule anytime for MTH 415 the previous week. I did end up studying a few hours on Sunday once I came back from spring break. This will be interesting to see what happens next as now school is entirely online. Since, I can no longer do anything outside of my home, I am expecting to have more time to work on school work.
\item How many hours did you participate during online office hours? Suggestions for improvement, if appropriate.
\vfill
None. I'll definitely be making use of this in the future.

\end{enumerate}

\newpage
The next few sections are for long-form responses. They involve working on your Mathematical Present, making connections with your Mathematical Past, and grasping towards your Mathematical Future.

\begin{enumerate}

\item {\large {\bf Active Reading}}:  \\
\\
I read over all of chapter 1 and 2 for the first exam. I should have also looked over Chapter 0 more than a simple skim. I am planning this week to do a deep dive into Chapter 3. I know I should have already been there before break even started, but I put other priorities first. I had skimmed over the chapter prior to Thursday's short/long quiz, but not much more than understand the main definitions. Just from that simple skim, I think chapter 3 will be very juicy. There are a lot of big ideas and connectors in this chapter.\\
\\
\item {\large {\bf Error-analysis}}:\\
\\
I need to work on mastering more proofs. Too often I say ``Oh, yeah. I got that. I could have figured that out on my own" and yet I never revist the problems that gave me troubles. Without revisiting I am not taking the information in as nearly as well as I could.\\
\\
On the exam, one of the questions was to prove Theorem 2.8:\\
 Let $ A $ be a toplogical space $ X $ and let $ A' $ be the set of limit point of $A$. Then $ Cl(A)=A\cup A' $\\
 This is the third encounter with this problem. Yet on exam day, with the pressure on I couldn't quite remember nor figure out how to best prove this. I differed from the proof in the book by giving cases for all sides. This can be done more succinctly by staying what must be true before splitting into cases.

\item {\large {\bf Consumers to Producers}}:  \\
\\
This reflection is more of a getting back to it deal. I've been under some crazy time scheduling and have been ignoring this part of the course. This is unacceptable. The biggest fault was starting this much too late. The day before or day of, isn't the best way to create a piece of quality work. But on the flip side never doing any in order to create the ``perfect" paper is also a poor way of improving. The more iterations I can do the more I can improve and see what went wrong/right.\\
\\
So this week is all about producing.\\
Before spring break, I had finally created physical flash cards for the first time ever. I've tested myself and done self quizzing, but nothing can quite beat a physical flash card. They're convenient to check from walking to and from classes, and can be an easy way to keep topology on the brain.\\
\\
I also did a lot of consuming before the big exam on Tuesday. I spent several days only focusing on topology (this was actually a bit detrimental to my Stats class). If an assignment wasn't due the next day, I wouldn't work on it so I could continue focusing on topology.

\item {\large {\bf Failing \& Succeeding}}:\\
\\
I failed at completing these reflections this semester. I did the first two, but they were no where near the expectations you had set. Since, I wasn't meeting them I decided to stop doing them in order to focus on other priorities. This was both a good idea and bad idea. While this helped me complete other work and get ahead in different classes, this is a bad way to work on being an active learner in topology.\\
\\
My biggest idea I need to work on is day by day progress. Too often my work goes as following: the first \%20 of the work during the first \%80 of the available time and then I finish \%80 of the work during the last \%20 of the given time. This works okay at best. The first \%20 lulls me into thinking I'm on top of it and have plenty of time. Then other stuff comes up and  I soon run out of time. If I work on it for a little each and every day I should have most, if not all of it, done by the due date. \\
\\
The hardest part is not getting thrown off at the first moment of distress. Every semester I start out working a little each day. And then the first assignment/exam comes and I focus on one area and get behind in others. This piggybacks into a negative feedback loop.
\\
%\vskip18cm
\item {\large {\bf Notes to Yourself}}: Anything else you'd like to add to this week's reflection for your attention next week.\\
\\
Really get to work this week. Since school is canceled, all you have to do is go to work and do school. This is your chance to really catch up and be ahead of the game rather than just in time. Take time every day to work on topology and the other objectives you have set.
\end{enumerate}


\end{document} 