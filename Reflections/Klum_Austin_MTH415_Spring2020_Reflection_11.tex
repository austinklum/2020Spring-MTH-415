\documentclass[11pt]{article}


\usepackage{amsmath, amssymb, amsthm, amscd, amsfonts, dsfont}
\usepackage{mathrsfs}
\usepackage{graphicx,array}
\usepackage{verbatim}
\usepackage{enumerate}
\usepackage{url,hyperref}
\usepackage{comment}
\usepackage{multicol}
\usepackage{tikz}
\usepackage{charter}
\usepackage{bbold}

\setlength\extrarowheight{6pt}

\newcommand{\C}{\mathbb{C}}
\newcommand{\R}{\mathbb{R}}
\newcommand{\Q}{\mathbb{Q}}
\newcommand{\Z}{\mathbb{Z}}
\newcommand{\N}{\mathbb{N}}

\renewcommand{\AA}{\mathcal{A}}
\newcommand{\BB}{\mathcal{B}}
\newcommand{\CC}{\mathcal{C}}
\newcommand{\DD}{\mathcal{D}}
\newcommand{\RR}{\mathcal{R}}
\renewcommand{\SS}{\mathcal{S}}
\newcommand{\TT}{\mathcal{T}}

\pagestyle{empty}
\thispagestyle{empty}
%%%%%%%%%%%%%%%%%%%%%%%%%%%%%%%%%%%%%%%%%%%%%%%%%%%%%%%%%%%%%%%%%%%%%
%                         ADJUST MARGINS
\usepackage[margin=0.67in]{geometry} % Makes margins 1in. Remove or comment out (with percent) if you want.

\begin{document}

\begin{center}
\Large \bfseries MTH 415 \textcolor{blue}{UPDATED} Weekly Reflection Activity Cover Sheet\\ 
{\it Due \underline{every Monday} by 11:30pm in Canvas}\\ 
\smallskip
\it Name/Section/Date/Submission \# \fbox{{\Large Austin Klum MTH-415 2020/04/27 }}
\end{center}

{\bf Effective learners} regularly self-assess their progress and then adjust study strategies in response. This reflection will help you to analyze your week's performance and help you on your path to improving your learning strategies.

\begin{enumerate}

\item What study techniques have you used in the last week:\\
	
	\begin{tabular}{|c|c|c|c|}
		\hline
		 {\bf Strategy} & {\bf Never} & {\bf Once} & {\bf Multiple (How many?)}\\
		\hline
		Identified and Removed Distractions While Studying &&X&\\
		\hline
		Studied in well-spaced blocks of time &&&3\\
		\hline
		Studied in blocks of time not exceeding 60 mins &&&2\\
		\hline
		Worked on Making/Taking Self-Timed Short Quizzes &X&&\\
		\hline
		\hline


		Studied Theorems and Definitions in Text&&&3\\
		\hline 
		Made Notes While Studying Text \& Online&&&2\\
		\hline 
		Worked on Error Analysis&&X&\\
		\hline 
		\hline
		 
		Studied and Reworked Exemplary Worked Examples in Text &X&&\\
		\hline 
		Studied and Reworked Examples/Proofs from Lecture Notes &X&&\\
		\hline 
		Studied and Reworked Worked Examples found Online &X&&\\
		\hline 
		\hline


		Made/Generated/Produced Questions this week &&X&\\
		\hline
		Answered Questions generated this week &X&&\\
		\hline
		\hline
		
		Online Office Hours with Dr. Das &&X&\\
		\hline		
		Asked Questions during Online Office Hours &X&&\\
		\hline
		\hline
		
		Other strategies (may describe in reflection) &&X&\\
		&&&\\
		\hline
		
			
	\end{tabular}

\item How many hours did you schedule for MTH 415 study this week? How many hours did you actually study this week? Suggestions for improvement, if appropriate.
\vfill
I spent close to 3 hours ``working" on topology. Of those 3 around 2 of them were intense and focused. I plan on setting 6-8pm my designated topology time at least three or four times a week.
\item How many hours did you participate during online office hours? Suggestions for improvement, if appropriate.
\vfill
I was not able to attend any this past week.
\end{enumerate}

\newpage
The next few sections are for long-form responses. They involve working on your Mathematical Present, making connections with your Mathematical Past, and grasping towards your Mathematical Future.

\section*{Active Reading:}

I spent some time reading chapter 6 on connectedness, some of its properties, and path connectedness. \\
\\
Connectedness is a big topic that manifests in topology. We only have one more 'C' word till we have the big three definitions of topology.\\
\\
The definition of connected is as follows: We say that a topology $ X $ is connected if there does not exists a pair of disjoint nonempty open sets whose union is $ X $.\\
\\
Connectedness is a topological property that carries between homeomorphisms. This is useful as this helps us work between topological spaces.\\
\\
Components are equivalence classes of connected subsets of $ X $. This is very similar to a partition, but slightly different. A partition would be on the actual set itself, where as the components are the connected components in the topology on $ X $.\\
\\
The chapter also spent time discussing the stronger definition of path connected. That is a topological space $ X $ is path connected if for every $ x,y \in X $ there exists a path in $ X $ from $ x $ to $ y $.\\
\\
Its interesting to see examples of topologies that are connected, yet are not path connected. This has some cool applications and uses.
\section*{Error-Analysis:}
I have made the error of not putting in enough practice time. Since school has gone online I've been struggling to keep up with everything. The hardest part is not feeling all that motivated in the sense of 'I  need to get this done to do well in school'. I'm much more inclined to say 'oh well' there's more than just school.\\
\\
One thing I need to keep straight is that being connected does not mean there exists a path between any two points. It is confusing at times to think about a topology that is connected, but does not have a path between all points.\\ 
\\
I need to practice drawing topologies and thinking about quotient spaces. They have been giving me troubles and I am getting frustrated with them.
\section*{Consumers to Producers:}
I added to my stack of definitions the new terms in section 6.1, 6.2, 6.3, and 6.4. This is useful as it allows me to self test myself and make sure I really know my definitions.\\
I need to be working on creating my writing projects. I have done very little thought into what I would like to write about.

\section*{Failing \& Succeeding}
I'd say this week was diffidently better than last weeks. I was able to enjoy my time more in the evenings and not feel so stressed out and stuck. I moved back into an apartment in La Crosse and it's been nice to see my buddies.\\
\\
So now that online classes are in full swing, I found that I really do not do well with them. I really miss the support system of talking with classmates and dropping in at office hours. I wish the semester was back to normal. \\
\\
I'm thinking I might need to reduce my hours at work. Or maybe work harder? I'm not sure. I feel like I still have plenty of free time (provided I manage my time well), but I am unable to make great progress on any of my homeworks. \\
\\
There just seems to be a lot of resistance and lack of momentum when working on all my assignments from home. I'm not sure what to do about it.\\


%\vskip18cm
\section*{Notes to Yourself:}
With your new job coming up you'll need to focus and work harder during the given times. It'll be a stressful few weeks here and I want you to push through it.\\
\\
Work harder in shorter bursts, rather than all day semi-working/looking at others things. If I can put in the needed effort and blast through any roadblocks this homework shouldn't take that long. I've felt like there have been a lack of direction and energy in my work. I miss others and do better when I can still see my friends. \\
\\
You're behind. Not in the sense you can't complete everything with more effort, but it's going to be a difficult week. Once we get through it, and you focus on what's important, we should be a much better spot.\\
\end{document} 