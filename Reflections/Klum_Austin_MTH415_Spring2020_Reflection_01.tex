\documentclass[10pt]{article}


\usepackage{amsmath}
\usepackage{graphicx,array}

\usepackage{amsmath}
\usepackage{graphicx}

\usepackage{charter}

\setlength\extrarowheight{6pt}


\pagestyle{empty}
\thispagestyle{empty}
%%%%%%%%%%%%%%%%%%%%%%%%%%%%%%%%%%%%%%%%%%%%%%%%%%%%%%%%%%%%%%%%%%%%%
%                         ADJUST MARGINS
\usepackage[margin=0.67in]{geometry} % Makes margins 1in. Remove or comment out (with percent) if you want.

\begin{document}

\begin{center}
\Large \bfseries MTH 415 Weekly Reflection Activity Cover Sheet\\ 
{\it Due \underline{every Monday} by 11:30pm in Canvas}\\ 
\smallskip
\it Name/Section/Date/Submission \# \fbox{Austin MTH-415 2020/02/17}
\end{center}

{\bf Effective learners} regularly self-assess their progress and then adjust study strategies in response. This reflection will help you to analyze your week's performance and help you on your path to improving your learning strategies.

\begin{enumerate}

\item What study techniques have you used in the last week:\\
	
	\begin{tabular}{|c|c|c|c|}
		\hline
		 {\bf Strategy} & {\bf Never} & {\bf Once} & {\bf Multiple (How many?)}\\
		\hline
		Identified and Removed Distractions While Studying &&&3 or 4\\
		\hline
		Studied in well-spaced blocks of time &&&3\\
		\hline
		Studied in blocks of time not exceeding 60 mins &&X&\\
		\hline
		Worked on Making/Taking Self-Timed Short Quizzes &X&&\\
		\hline
		\hline
		Read Text/Notes Before Class Lesson &&&2\\
		\hline
		Read Text/Notes After Class Lesson &&&2\\
		\hline
		\hline

		Studied theorems and definitions in text&&&3\\
		\hline 
		Made my own notes&&&2\\
		\hline 
		Worked on error analysis&&X&\\
		\hline 
		\hline
		 
		Studied and Reworked Exemplary Worked Examples in Text &&&3\\
		\hline 
		Studied and Reworked Examples/Proofs from Lecture Notes &&X&\\
		\hline 
		Studied and Reworked Exemplary Quiz Solutions &X&&\\
		\hline 
		Studied and Reworked Worked Examples found Online &X&&\\
		\hline 
		\hline
		
%		YouTube, Khan Academy, Paul's Online Math Notes, etc.&&&\\
%		\hline
%		Tutoring Center at Murphy&&&\\
%		\hline

		Office Hours with Dr. Das &&&2.5\\
		\hline		
		Group Study&&X&\\
		\hline
		\hline
		Asked Questions in Class &&X&\\
		\hline
		Answered Questions in Class &&X&\\
		\hline
		\hline
		
		Made/Generated/Produced Questions in Class &&X&\\
		\hline
		Made/Generated/Produced Questions during Study &&X&\\
		\hline
		Answered Questions generated in Class or during Study &X&&\\
		\hline
		\hline
		
		Other strategies (may describe in reflection) &&&3\\
%		&&&\\
		\hline
		
			
	\end{tabular}

\item How many hours did you schedule for MTH 415 study this week? How many hours did you actually study outside of Lectures this week. Suggestions for improvement, if appropriate.
\vfill
I schedule 3 hours on Saturday and 8 hours on Sunday. On Monday, I worked on Topology for 3 hours. I need to be more effective with that study time.

\item How many hours did you see Dr. Das outside of lectures this week? Suggestions for improvement, if appropriate.
\vfill
I saw Dr. Das for ~2 hours outside lecture during office hours. While this was alright, I want to bring this closer to 3 or 4 hours a week.

\end{enumerate}

\newpage
The next few sections are for long-form responses. They involve working on your Mathematical Present, making connections with your Mathematical Past, and grasping towards your Mathematical Future.

\begin{enumerate}

\item {\large {\bf Active Reading}}:  
\begin{itemize}
\item Remember Tom Forde's advice: ``{\bf Read with paper and pen}. As you are reading through the text, you should be writing notes and verifying any parts of which you are skeptical. Check any calculations. Rewrite definitions and theorems in your own words.'' Rework problems. Generate your own problems by {\it variation}. Write down notes summarizing key content from your active reading this week.

\item What did you find neat/cool/beautiful/incredible/\dots while you were studying the textbook this week?

\item Strive to discover connections with the math from your past. Here are sample sentence starters: ``This reminds me of \dots from class \dots \dots because \dots''  and ``I used to think \dots but now I think \dots because \dots''

\item Write down notes summarizing the key content from your studies this week.	
	
\end{itemize}

\item {\large {\bf Error-analysis}}:
Always be on the hunt for errors! Don't let them slip away, or brush them under a rug!
Find them, dissect them, and analyze what went wrong. Engage deeply rather than superficially with your errors.
		\begin{itemize}
			\item Write down the problems (aim to catch {\bf at least 5 per week}).							\item Identify your mistake(s) or misconception(s). 
			\item Try to analyze why it was natural to make such a mistake in the first place.
			\item Rework the problem(s) correctly.		
		\end{itemize}

\item {\large {\bf Consumers to Producers}}:  

\begin{itemize}
\item What did you produce this week? How did it compare with what you consumed? 
\item Generate questions/conjectures based on your active engagement with the text, my notes, other sources. Apart from questions directly related to the math of your present week, some of your questions may reach back to the math of your past, others may be grasping at the math of your future!
\item Try and answer some of them! 	There are some questions you may be able to answer under 60 minutes, others may take 60 months, or even more!
\end{itemize}

\item {\large {\bf Failing \& Succeeding}}:
Too often failure and struggle is treated very negatively, but failing and struggling is one of the most important aspects of learning. If treated appropriately, failure can be your friend and motivate you to do better. Write about mathematical failures that you have experienced this week, how you overcame the failures, and what your learned from the failures. How about mathematical successes that you had this week? Save these writings, they will help you compose a final reflection that will be due at the end of the term.

%\vskip18cm
\item {\large {\bf Notes to Yourself}}: Anything else you'd like to add to this week's reflection for your attention next week.

\end{enumerate}

\hrulefill

\medskip
{\Large 
~{\it General instructions (please visit for more feedback on your individual reflections):}
}

\begin{itemize}

\item I strongly recommend that you work on improving your LaTeX skills, especially for typing up mathematics that you want to save now and be able to access easily in the future. This is a skill that you can put on your resume, and it will help you showcase the mathematics that you have produced for the benefit of future employers (including graduate schools).

\item There is no upper bound but each reflection should have at least 5 pages beyond the cover sheet. You should be working on some 415 every day, at the end of which you should consider what could go in your reflections. If you do this for about 15 minutes each day you will find that you are making great incremental progress towards your weekly reflection submission. 

\item Your reflection may involve various parts from various places (e.g., you have worked on some error analysis in one notebook, but have made your notes for a specific section of the text on loose leaf).

\item If you find that you have not been able to LaTeX some parts of your reflection, you may scan them and include them in your submission. This will be especially useful for diagrams, You all have access to free scanning at Murphy Library. Also, there are a number of free pdf scanners that allow you to take pictures using a phone and then covert the image to a pdf.

\item Your weekly submission must be {\bf one pdf}  appropriately labelled using the filename format:\\
\phantom{XXXXXXXXXXXXXXXXXX}{\bf LastName\_FirstName\_MTH\_415\_S2020\_Reflect\_YY.pdf}\\
YY is the submission number (e.g. the first submission which is due on Monday 2/17 will be numbered 01)



\end{itemize}

\newpage

\section{Active Reading}

As I was reading, I was redoing the book into a tool called Notion. Notion is a note taking/organizer software similar in nature to Evernote. I started using it at the end of last semester to help connect all the ideas and work I have done. So as I read the text, I make sure to transcribe the ``important" information and my own interpretation on it.\\
\\
 I make sure I read the text, and then from memory try to rewrite it into my note taking system. This is my first attempt at recall. This is done several more times throughout the week to ensure I remember the definitions. I put in theorems and have a toggle with the proof hidden underneath. I also make sure to redo the proof on my own and only look at the textbook's version when I cannot make any more progress. When reviewing my notes, I can attempt to again recall the general outline of the proof and see how close I actually was. \\
\\
One thing I need to make sure about is doing this for all text in the book. I take detailed notes on a lot of it, but due to time constraints and priorities I am not able to take detailed notes on everything. This is especially true these past few days where I have felt more behind than normal.\\
\\
I thought bases made much more sense now than when I first learned of them last spring. Last spring I didn't really understand how they connected to Topologies and their set. I just understand that using these bases we could generate topologies. All I needed to ``know" was how to apply them to the current problem. I understand them much more clearly as just a collection of open sets. In fact it can contain elements not found in the topology. Using these open sets you can then create a topology from these. All a basis element is an open set. If I took an open set in a topological space X and a basis element that described the same set. I would find that they are equivalent and that the open set is also a basis element.\\
\\
 I think that its really interesting/frustrating that the term basis is used so much in Mathematics. In Linear Algebra, we had bases that generated vector spaces. Yet these bases are very different from the bases found in Topology. They all share the idea of having a smaller function/descriptor that can generate whole objects. I just wish the method/form of the basis was more similar. \\
\\
I am trying to relate my Computer Science background to Mathematics as relating new material to what you already know helps cement and provided connections and ideas. I've been trying to think of ways to describe the elements of a Topology. As a Topology is a collection of open sets, each element is a set in itself. This is really interesting as elements are sets. Within that particular element, you can find more elements that belong to that element. This reminds me of having a list of lists in Computer Science. As there is a list, the open sets in the topology, and within each of those sets we find another list, the elements inside the open set. It helps me to think of having a cabinet drawer and each time you open a draw you find another cabinet that contains the items.\\
\\
In a way bases can kinda be described as constructors in Computer Science. When a new object is created, say a topology, a constructor function is ran that helps create and setup the new object. This constructor is very similar to a basis as it is a smaller way of representing many objects from the same class. 

\newpage

\section{Error-Analysis}
In Theorem 1.9: Let $X$ be a set and $B$ be a basis for a topology on $X .$ Then
$U$ is open in the topology generated by $\mathcal{B}$ if and only if for each $x \in U$ there exists a basis element $B_{x} \in \mathcal{B} \text { such that } x \in B_{x} \subset U$. 
\\
I misunderstood and took a while to understand in particular the line:
\[\ldots  \text { such that } x \in B_{x} \subset U \]
This was interesting as this really mean $ x\in B_x $ \textit{\bf and} $ B_x \subset U $. I feel like this is a little bit of an abuse of notation as $ B_x \subset U$ feels like a non-sequitur. While it does make the idea more compact and hence ``cleaner", I feel like this is misleading for the reader. \\
\\
If I go back to Computer Science, the best code isn't the one with the fewest lines. The best code is instead the code which you or someone else will understand quickly and easily 6 months from now. Compacting ideas into neat little packages feels like it is more clear in a ``less is more approach", but this compaction makes understanding the ideas much more difficult later on. Since, the reader now has to unpack the idea to understand the true meaning it causes some dissonance between the reader and ideas presented.

\newpage

\section{Consumers to Producers}
Much of this week was spent consuming. It is far easier and less effortful to consume than to produce. Producing takes time and is work. Especially since all prior Mathematics and knowledge gaining in school has been primarily consumption based. \\
\\
There's nothing inherently wrong with just consuming. Consuming is often times good enough for many applications of life. But to be truly great and completely grok an idea, concept, or pattern, one needs to produce. This is especially vital in the study of Mathematics. \\
\\
To truly understand an idea in Mathematics, one has to wrestle with it. They must go on to use and apply the idea in situations. They need to connect this new idea with what is already known. By doing this they will be increasing the connections and further cementing the idea into place.\\
\\ 
What I did produce was a way to help explain the idea of Coarser and Finer to myself. I call it the Hershey Bar topology. If you think of a hershey bar, you have an entire set X. The open sets in this hershey bar set are the little bars of chocolate that you can break the full bar into. \\
\\
Using this to explain Coarser and Finer we can say the following. Think of a Hershey bar. It's has 12 (4x3) squares comprising the whole set.\\
\\
If we break up the bar into all 12 squares, this is the finest it can be. This is the most broken up/discrete/ smallest parts way we can have the bar. If we leave the bar as one big square, this is the coarsest we can have the bar.\\
\\
Suppose we wanted pocket chocolate for a snack later. Which set would be easier to put in your pocket? The finer one. Which is the one where all the squares are broken into its component parts. A whole bar would be awkward as it's bulky and coarse.\\
\\
While it may be a silly example, I have been using it to help me think about comparing two topologies. I actually think the silliness of the example is why I remember it so. Especially since the gravel and sand explanation doesn't really account for the idea of getting more gravel or sand. There really isn't a limit on the size or the amount of gravel or sand. By restricting these I find myself better understanding the underlying intuition behind Coarser and Finer.\\
\\
I idea I have really been toying with is the connection of Topolgical space to real analysis. While I haven't taken the course, from what I heard and my basic understanding is that real analysis is essentially an applied version of Topology. We limit ourself to $ R^n $, yet many of the same ideas apply to both fields. I would love to learn more about how they are connected.\\

\newpage
\section{Failing \& Succeeding}

I feel like I am understanding the material much better as whole than last year when I took this course with Dr. George. I felt like I was completely lost for most of the semester. Each assignment I was faced with a struggle. While struggles are normally good and are a good indicator (cause?) growth. Struggling helps cement ideas and shows you where you're blind spots are. Without this struggle, much of mathematics will not stick pass the time line of an upcoming exam.\\
\\
Yet this was an unproductive struggle. I spent hours and hours staring at the paper, practically getting no where. Unproductive Struggle is incredibly frustrating, as it's a waste of time. I do not grow from the encounter and the work is still not completed.  I would scour the internet looking for help, but could not find much. I went to office hours multiple times a week and I would talk with friends about the problems. I specifically went on Sundays to the math tutoring center so that way Collin would be able to help guide me in the homework. (Colin was also taking the class at the same time as myself)\\
\\
Even with this ``Success" I still have a lot to do. I have done poorly on all my ``daily blood check up" quizzes. This is again interesting as I am close to understanding and feel like the answer is on the tip of my tongue. I feel like I am often times close to figuring out the proof, but I am missing some idea or exact wording that I can't double check during the quiz times to fully articulate the idea I want to use. 

\newpage
\section{Notes to Myself}
Alright Austin. You need to bust butt more.\\
\\
I start out good on the weekends, spending several hours spaced out over the course of the day to work on the material. Yet throughout the rest of the week I do not work much on Topology. This is a poor way of doing it as it slowly makes me more and more behind. We are on week 3 or 4 now and I am now behind in my notes. I was a little behind the prior week and mostly on top of it the week prior to that. But little by little, I am now behind the material being taught in class.\\
\\
I need to work a little bit on Topology each and every day. Even spending 20 minutes before bed would be immensely beneficial to my understanding and doing well in this course. The biggest hurdle to doing so is being tired and busy with other obligations such as Work, Dance, and other classes. \\
\\
The hardest part for me is that I am taking 10 undergradate credits, 6 graduate credits, working 18.5 hours a week, teaching dance lessons two nights a week for two hours each time, and hosting a social dance every Friday.\\
\\
I have a hard time prioritizing all of it. I hate working all day, only to have to work all night. I am an incredibly social person and really just want to hang out and relax with others. I need this ``downtime" to help function.


\end{document} 