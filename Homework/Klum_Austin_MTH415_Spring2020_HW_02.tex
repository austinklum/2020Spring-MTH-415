%This is a Latex file.
\documentclass[12pt]{article}

\usepackage{amsmath, amssymb, amsthm, amscd, amsfonts, dsfont}
\usepackage{mathrsfs}
\usepackage{graphicx}
\usepackage{verbatim}
\usepackage{enumerate}
\usepackage{url,hyperref}
\usepackage{comment}
\usepackage{multicol}
\usepackage{latexsym,fancyhdr}
\usepackage[margin=1in]{geometry}
\usepackage{lastpage} % Required to determine the last page for the footer
\usepackage{tikz}
\usepackage{charter}
\usepackage{bbold}

\parindent 0pt

\pagestyle{fancy} \lhead{\sf MTH 415} \chead{\sf Homework \#2}
\rhead{\sf Due: Tuesday 02/25/2020} \lfoot{} \cfoot{} \rfoot{}

\newcommand{\C}{\mathbb{C}}
\newcommand{\R}{\mathbb{R}}
\newcommand{\Q}{\mathbb{Q}}
\newcommand{\Z}{\mathbb{Z}}
\newcommand{\N}{\mathbb{N}}

\renewcommand{\AA}{\mathcal{A}}
\newcommand{\BB}{\mathcal{B}}
\newcommand{\CC}{\mathcal{C}}
\newcommand{\DD}{\mathcal{D}}
\newcommand{\RR}{\mathcal{R}}
\renewcommand{\SS}{\mathcal{S}}
\newcommand{\TT}{\mathcal{T}}

\newcommand{\wts}[1]{\textit{\textcolor{blue}{WTS: #1}}\\}
\newcommand{\pp}{\textit{\textcolor{yellow}{PP: }}}%[1]{\textit{\textcolor{yellow}{PP: #1}}\\}


\begin{document}
\begin{enumerate}
	
	\item[\pp1.25] Prove that, in a topological space $X,$ if $U$ is open and $C$ is closed, then $U-C$ Is open and $C-U$ is closed.
	
	\item[\pp1.26] Prove that closed balls are closed sets in the standard topology on $\mathbb{R}^{2}$.
	
	\item[\pp1.27] The infinite comb $C$ is the subset of the plane illustrated in Figure 1.17 and defined by
	\[C=\{(x, 0) | 0 \leq x \leq 1\} \cup\left\{\left(\frac{1}{2^{2}}, y\right) | n=0,1,2, \ldots \text { and } 0 \leq y \leq 1\right\}\]
	(a) Prove that $C$ is not closed in the standard topology on $\mathbb{R}^{2}$.
	
	(b) Prove that $C$ is closed in the vertical interval topology on $\mathbb{R}^{2}$.
	\item[\pp1.33] Prove theorem 1.17: Let $ X $ be a topological space.
	\begin{enumerate}
		\item[(a)]Prove that $\varnothing$ and $X$ are closed sets.\\
		\item[(b)]Prove that the intersection of any collection of closed sets in $X$ is a closed set\\
		\item[(c)]Prove that the union of finitely many closed sets in $X$ is a closed set.\\
	\end{enumerate}
	\item[\pp1.35] Show that $\mathbb { R }$ in the lower limit topology is Hausdorff.\\
	
	\item[\pp1.36]  Show that $\mathbb { R }$ in the finite complement topology is not Hausdorff.
	
	\item[\pp2.02] Prove theorem 2.2: Let $ X $ be a topological space and $ A $ and $ B $ be subsets of $ X $.
	\begin{enumerate}
		\item[(a)]If $C$ is a closed set in $X$ and $A \subset C ,$ then $\mathrm { Cl } ( A ) \subset C$\\
		
		\item[(b)] If $A \subset B$ then $\mathrm { Cl } ( A ) \subset \mathrm { Cl } ( B )$\\
		
		\item[(c)] $A$ is closed if and only if $A = \mathrm { Cl } ( A )$\\
		
	\end{enumerate}
	\item[\pp2.07] Let $B=\left\{\frac{a}{2^{n}} \in \mathbb{R} | a \in \mathbb{Z}, n \in \mathbb{Z}_{+}\right\} .$ Show that $B$ is dense in $\mathbb{R}$
	
	\item[2.10] Prove Theorem 2.5: Let $X$ be a topological space, $A$ be a subset of $X,$ and $y$ be an element of $X .$ Then $y \in \mathrm{Cl}(A)$ if and only if every open set containing $y$ intersects $A$.
	\begin{proof}
		Let $ (X,\TT) $ be a topological space, $ A\subset X $ and $ y\in X $.\\
		($ \Rightarrow $) \wts{If $ y\in Cl(A) $ then $ \forall U \in \TT $ such that $ y\in U $ we will have the following $ U\cap A \not= \varnothing $}
		By way of contradiction, suppose $ y\not\in Cl(A) $. Then there exists a closed set $ C $ such that $ y\not\in C$. Thus, $ X-C $ is open and $ y\in X-C \subset X-A $. Notice, $ (X-C)\cap A = \varnothing $. This is a contradiction as $ X-C $ is an open set containing $ y $, yet $ (X-C)\cap = \varnothing $. 
		Therefore, if $ y\in Cl(A) $ then every open set containing $y$ intersects $A$.\\
		\\
		($\Leftarrow$) \wts{If $ \forall U \in \TT  $ such that $ y\in U $ and $ U\cap A \not= \varnothing $ then $ y\in Cl(A) $}
		 Let $U$ be open and $ U\cap A \not= \varnothing $. By theorem 2.4, We have that $ U \in Int(A) $. Recall, by definition $ Int(A) \subset A \subset Cl(A) $. Since, $ y\in U \in Int(A) $ it follows that $ y\in Cl(A) $\\
		Therefore $ \forall U \in \TT  $ such that $ y\in U $ and $ U\cap A \not= \varnothing $ then $ y\in Cl(A) $
	\end{proof}
	
	\item[\pp2.11] Prove Theorem 2.6: For sets $ A $ and $ B $ in a topological space $ X $, the following hold:
	\begin{enumerate}
		\item[(a)] $\mathrm { Cl } ( X - A ) = X - \operatorname { Int } ( A )$\\
		\item[(b)] $\operatorname { Int } ( A ) \cap \operatorname { Int } ( B ) = \operatorname { Int } ( A \cap B )$\\
	\end{enumerate}
	
	
	
	\item[2.13] Determine the set of limit points of $ A $ in each case.
		\begin{enumerate}
			\item[(a)] $A=(0,1]$ in the lower limit topology on $\mathbb{R}$.
			\item[(b)] $A=\{a\}$ in $X=\{a, b, c\}$ with topology $\{X, \varnothing,\{a\},\{a, b\}\}$
			\item[(c)] $A=\{a, c\}$ in $X=\{a, b, c\}$ with topology $\{X, \varnothing,\{a\},\{a, b\}\}$
			\item[(d)] $A=\{b\}$ in $X=\{a, b, c\}$ with topology $\{X, \varnothing,\{a\},\{a, b\}\}$
			\item[(e)] $\quad A=(-1,1) \cup\{2\}$ in the standard topology on $\mathbb{R}$
			\item[(f)] $A=(-1,1) \cup\{2\}$ in the lower limit topology on $\mathbb{R}$
			\item[(g)] $A=\left\{(x, 0) \in \mathbb{R}^{2} | x \in \mathbb{R}\right\}$ in $\mathbb{R}^{2}$ with the standard topology.
			\item[(h)] $A=\left\{(0, x) \in \mathbb{R}^{2} | x \in \mathbb{R}\right\}$ in $\mathbb{R}^{2}$ with the topology generated by the basis in Exercise 1.19
			\item[(i)] $\quad A=\left\{(x, 0) \in \mathbb{R}^{2} | x \in \mathbb{R}\right\}$ in $\mathbb{R}^{2}$ with the topology generated by the basis Exercise $1,19$
		\end{enumerate}
	\item[2.20] Prove Theorem 2.11 : Let $A$ be a subset of $R^{n}$ in the standard topology. If $x$ is a limit point of $A,$ then there is a sequence of points in $A$ that converges to $x$.
	\begin{proof}
		
	\end{proof}
	\item[\pp2.21] Determine the set of limit points of the set
	$$
	S=\left\{\left(x, \sin \left(\frac{1}{x}\right)\right) \in \mathbb{R}^{2} | 0<x \leq 1\right\}
	$$
	as a subset of $\mathbb{R}^{2}$ in the standard topology. (The closure of $S$ in the plane is known as the topologist's sine curve.)
	
\end{enumerate}
\end{document}

