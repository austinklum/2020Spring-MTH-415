%This is a Latex file.
\documentclass[12pt]{article}

\usepackage{amsmath, amssymb, amsthm, amscd, amsfonts, dsfont}
\usepackage{mathrsfs}
\usepackage{graphicx}
\usepackage{verbatim}
\usepackage{enumerate}
\usepackage{url,hyperref}
\usepackage{comment}
\usepackage{multicol}
\usepackage{latexsym,fancyhdr}
\usepackage[margin=1in]{geometry}
\usepackage{lastpage} % Required to determine the last page for the footer
\usepackage{tikz}
\usepackage{charter}
\usepackage{bbold}

\parindent 0pt

\pagestyle{fancy} \lhead{\sf MTH 415} \chead{\sf Homework \#2}
\rhead{\sf Due: Tuesday 02/25/2020} \lfoot{} \cfoot{} \rfoot{}

\newcommand{\C}{\mathbb{C}}
\newcommand{\R}{\mathbb{R}}
\newcommand{\Q}{\mathbb{Q}}
\newcommand{\Z}{\mathbb{Z}}
\newcommand{\N}{\mathbb{N}}

\renewcommand{\AA}{\mathcal{A}}
\newcommand{\BB}{\mathcal{B}}
\newcommand{\CC}{\mathcal{C}}
\newcommand{\DD}{\mathcal{D}}
\newcommand{\RR}{\mathcal{R}}
\renewcommand{\SS}{\mathcal{S}}
\newcommand{\TT}{\mathcal{T}}

\newcommand{\wts}[1]{\textit{\textcolor{blue}{WTS: #1}}\\}
\newcommand{\pp}{\textit{\textcolor{yellow}{PP: }}}%[1]{\textit{\textcolor{yellow}{PP: #1}}\\}


\begin{document}
	\begin{enumerate}
		
		\item[1.25] Prove that, in a topological space $X,$ if $U$ is open and $C$ is closed, then $U-C$ Is open and $C-U$ is closed.
		\begin{proof}
		Let $ X $ be a topological space with $ U $ being open and $ C $ being closed.\\
		Notice, $ U-C $ can be rewritten to $ U\cap C^{\complement} $. By definition of closed set we have that the complement is open. Thus, as $ U $ is open and finite intersections of open sets are open we must have that $ U-C $ is open.\\
		Notice, $ C-U $ can be rewritten to $ C\cap U^{\complement} $. By definition of open set we have that the complement is open. Thus, as $ C $ is closed and arbitrary unions of closed sets are closed we must have that $ C-U $ is closed.
		\end{proof}
		\item[1.26] Prove that closed balls are closed sets in the standard topology on $\mathbb{R}^{2}$.\\
		\begin{proof}
		If the closed ball $ \BB(x,\epsilon)$ is equal to X, then $ \BB $ must be closed as the complement is the empty set which by definition must be open. Thus, $ \BB $ must be closed. \\
		Suppose that the closed ball $ \BB $ is not equal to $ X $ and hence not the empty set.
		Then there must exist an element $ y\in \BB^\complement $. Let the $ d(x,y) = h > \epsilon $.\\
		\wts{$\BB'(y,h-\epsilon) \subset \BB^\complement$}
		Suppose by way of contradiction, the open ball $\BB'(y,h-\epsilon) \not\subset \BB^\complement$. Then, there exists a $ z\in \BB' $ such that $ z\in\BB $.\\
		Notice, $ d(x,y)\leq d(x,z)+d(z,y) $. Which gives us $ d(x,z)\leq \epsilon$, $d(z,y) < h-\epsilon \Rightarrow r+d(z,y)<h-\epsilon+\epsilon<h, $ and $ d(x,z)+d(z,y)<h$.\\
		Then by transitivity, we have that $d(x,y)<h$. But this is a contradiction as $d(x,y)=h$. Thus, we have that $ z \not\in \BB' $ which then gives us $ z\in \BB^\complement $. Hence, $ \BB' \subset \BB^\complement $. And so, $ z $ is an interior point of $ \BB^\complement $ and implies that $ \BB^\complement $ must be open.\\ Therefore, by definition of closed we must have that $ \BB $ is closed.
		\end{proof}
		\item[1.27] The infinite comb $C$ is the subset of the plane illustrated in Figure 1.17 and defined by
		\[C=\{(x, 0) | 0 \leq x \leq 1\} \cup\left\{\left(\frac{1}{2^{2}}, y\right) | n=0,1,2, \ldots \text { and } 0 \leq y \leq 1\right\}\]
		(a) Prove that $C$ is not closed in the standard topology on $\mathbb{R}^{2}$.
		\wts{$\R^2-C$ is not open}
		Consider the point $ p=(0,1/2)\in \R^2-C$ and ball centered at point $p$ with radius $ \epsilon>0 $. We can find some $ \frac{1}{2^n}<\epsilon$. Hence, the point $ (\frac{1}{2^n},\frac{1}{2}) $ has distance to $ p $ less than $ \epsilon $. Thus, we have $ 0 $ as a limit point of $ C $, but $ (0,\frac{1}{2}) \not\in C$. Hence, $ C $ is not closed in the standard topology on $ \R^2 $.
		(b) Prove that $C$ is closed in the vertical interval topology on $\mathbb{R}^{2}$.
		\[\text{RECALL: Vertical Interval Topology is generated by } \{\{a\}\times(b,c)\subset \R^2| a,b,c\in \R\}\}\]
		\wts{$\R^2-C$ is open}
		
		\item[1.33] Prove theorem 1.17: Let $ X $ be a topological space.
		\begin{proof}
		\begin{enumerate}
			\item[(a)]Prove that $\varnothing$ and $X$ are closed sets.\\
			Notice, $ \varnothing,X \subseteq X $ and $ X-\varnothing = X $. Thus, as $\varnothing$ is open,$X$ is closed. Similarly, $ \varnothing - X = \varnothing $. Thus, $ \varnothing $ and $ X $ are closed sets.
			\item[(b)]Prove that the intersection of any collection of closed sets in $X$ is a closed set\\
			Let $ \cap_{i\in I} U_i $ be the intersection of a indexed collection of closed sets of $ X $. Taking the complement, we have $\cap_{i\in I} X-U_i  $. Since, each $ U_i $ is closed for each $ i\in I $, we have $ X-U_i $ is open for each $ i\in I $. Thus, we have a intersection of arbitrary open sets which is open. Therefore, $ \cap_{i\in I} U_i $ is closed. The intersection of any collection of closed such that in $ X $ is a closed set.
			\item[(c)]Prove that the union of finitely many closed sets in $X$ is a closed set.\\
			Let $ \cup_{i=1}^n U_i $ be a union of a finite number of closed sets in $X$. Taking the complement, we have $ \cap_{i=1}^n X-U_i $. Since, each $ U_i $ is closed, we must have $X-\cup_i $ is open. Therefore, the complement is a finite union of open sets, which is open. Thus, $ \cup_{i=1}^n U_i $ is closed. Thus, the union of finitely many closed sets in $ X $ is a closed set.
		\end{enumerate}
		\end{proof}
		\item[1.35] Show that $\mathbb { R }$ in the lower limit topology is Hausdorff.\\
		\begin{proof}
		Suppose $ a,b $ are distinct point in the lower limit topology. Assume without loss of generality, $ a<b $. Notice, $ [a,b) $ and $ [b,b+1) $ are disjoint open neighborhoods of $ a $ and $ b $. Therefore, $\R$ is Hausdorff in the lower limit topology.
		\end{proof}

		\item[1.36]  Show that $\mathbb { R }$ in the finite complement topology is not Hausdorff.\\
		\begin{proof}
				By way of contradiction, suppose $ U,V $ are disjoint open sets. Then, $ V \subset (\R-U) $. Notice, $ \R -U $ is a finite set and so $ V $ is finite. But $ \R-V $ is infinite, which is a contradiction to $ V $ is open.\\
				Therefore, $\mathbb { R }$ in the finite complement topology is not Hausdorff.
				\end{proof}
		\item[2.02] Prove theorem 2.2: Let $ X $ be a topological space and $ A $ and $ B $ be subsets of $ X $.
		\begin{enumerate}
			\item[(a)]If $C$ is a closed set in $X$ and $A \subset C ,$ then $\mathrm { Cl } ( A ) \subset C$\\
			Let $ C $ be a closed set in $ X $ and $ A\subset C $. Notice the $ Cl(A) $ is the finite intersection of closed sets. Thus, $ Cl(A) \subset C$ as $ C $ is either the smallest closed set in the intersection or larger than the $ Cl(A) $. 
			\item[(b)] If $A \subset B$ then $\mathrm { Cl } ( A ) \subset \mathrm { Cl } ( B )$\\
			Let $ A \subset B $. Notice, that the $ Cl(A) $ and $ Cl(B) $ are the smallest closed sets containing $ A $ and $ B $ respectively. Since, $ A $ is contained in $ Cl(B) $ we must have that $ Cl(A)\subset Cl(B)$
			\item[(c)] $A$ is closed if and only if $A = \mathrm { Cl } ( A )$\\
			Assume $ A $ is closed. Then, $ A $ is in the intersection of all closed sets and $ Cl(A) $ is the smallest closed set containing $ A $. Thus, the intersection will be equal to $ A $. \\
			Therefore $ A=Cl(A) $ \\
			\\
			Suppose $ A=Cl(A) $. Notice, $ Cl(A) $ is closed as the finite intersection of closed sets is closed. \\
			Therefore, $ A $ is closed.
		\end{enumerate}
		
		\item[2.07--] Let $B=\left\{\frac{a}{2^{n}} \in \mathbb{R} | a \in \mathbb{Z}, n \in \mathbb{Z}_{+}\right\} .$ Show that $B$ is dense in $\mathbb{R}$\\
		\begin{proof}
		Let $ \epsilon > 0 , x\in \R $, and $ a_1,a_2\in B $ such that $ a_1<x<a_2 $. Define $ a_1 := \frac{m-1}{2^n} $ and $ a_2:=\frac{m+1}{2^n} $. Notice, $ \frac{m-1}{2^n} < x < \frac{m+1}{2^n} $ implies $ \frac{-1}{2^n} < x - \frac{m}{2^n} < \frac{1}{2^n}$. We can then manipulate this further for $ |x-\frac{m}{2^n} | < \frac{1}{2^n} < \epsilon$. Thus, we have $ Cl(B) = \R $\\
		Therefore, $ B $ is dense in $ \R $.
		\end{proof}
%		https://www.math.ucdavis.edu/~npgallup/m17_mat25/lecture_notes/lecture_17/m17_mat25_lecture_17_notes.pdf\\

		
		\item[2.10] Prove Theorem 2.5: Let $X$ be a topological space, $A$ be a subset of $X,$ and $y$ be an element of $X .$ Then $y \in \mathrm{Cl}(A)$ if and only if every open set containing $y$ intersects $A$.
		\begin{proof}
			Let $ (X,\TT) $ be a topological space, $ A\subset X $ and $ y\in X $.\\
			($ \Rightarrow $) \wts{If $ y\in Cl(A) $ then $ \forall U \in \TT $ such that $ y\in U $ we will have the following $ U\cap A \not= \varnothing $}
			By way of contradiction, suppose $ y\not\in Cl(A) $. Then there exists a closed set $ C $ such that $ y\not\in C$. Thus, $ X-C $ is open and $ y\in X-C \subset X-A $. Notice, $ (X-C)\cap A = \varnothing $. This is a contradiction as $ X-C $ is an open set containing $ y $, yet $ (X-C)\cap = \varnothing $. 
			Therefore, if $ y\in Cl(A) $ then every open set containing $y$ intersects $A$.\\
			\\
			($\Leftarrow$) \wts{If $ \forall U \in \TT  $ such that $ y\in U $ and $ U\cap A \not= \varnothing $ then $ y\in Cl(A) $}
			Let $U$ be open and $ U\cap A \not= \varnothing $. By theorem 2.4, We have that $ U \in Int(A) $. Recall, by definition $ Int(A) \subset A \subset Cl(A) $. Since, $ y\in U \in Int(A) $ it follows that $ y\in Cl(A) $\\
			Therefore $ \forall U \in \TT  $ such that $ y\in U $ and $ U\cap A \not= \varnothing $ then $ y\in Cl(A) $
		\end{proof}
		
		\item[2.11] Prove Theorem 2.6: For sets $ A $ and $ B $ in a topological space $ X $, the following hold:
		\begin{enumerate}
			\item[(a)] $\mathrm { Cl } ( X - A ) = X - \operatorname { Int } ( A )$\\
			If we take the complement of both sides we have the following:
			\[X-Cl(X-A)=Int(A)\]
			Notice, a point $x \not\in Int(A) \Leftrightarrow \forall U\in \TT $ such that $ x\in U $, $ U\not\subset A $ and $ U\cap (X-A)\not= \varnothing \Leftrightarrow x\in Cl(X-A)$.\\
			Thus, $ x \in Int(A) \Leftrightarrow x\in X-Cl(X-A) $.\\
			Therefore, $ Cl(X-A)=X-Int(A) $
			\item[(b)] $\operatorname { Int } ( A ) \cap \operatorname { Int } ( B ) = \operatorname { Int } ( A \cap B )$\\
			 Let $ n\in Int(A) \cap Int(B) $. Then $ n\in Int(A) $ and $ n\in Int(B) $. Which gives us that $ A $ is a neighborhood of $ n $ and $ B $ is a neighborhood of $ n $. Thus, $ A\cap B $ is a neighborhood of $ n $. Hence, $ n \in Int(A\cap B) $. \\
			Therefore, $ Int(A)\cap Int(B) \subset Int(A\cap B) $\\
			\\
			Let $ n \in Int(A\cap B) $. Then, $ A\cap B $ is a neighborhood of $ n $. It follows that $ n\in Int(A) $ and $ n\in Int(B) $. Thus, $ n\in Int(A) \cap Int(B) $.\\
			Therefore, $ Int(A\cap B) \subset Int(A) \cap Int(B) $,\\
			\\
			Therefore, $ Int(A)\cap Int(B) = Int(A\cap B) $\\
		\end{enumerate}
		
		
		
		\item[2.13] Determine the set of limit points of $ A $ in each case.
		\begin{enumerate}
			\item[(a)] $A=(0,1]$ in the lower limit topology on $\mathbb{R}$.\\
			$ A'=[0,1) $
			\item[(b)] $A=\{a\}$ in $X=\{a, b, c\}$ with topology $\{X, \varnothing,\{a\},\{a, b\}\}$\\
			$ A' = \{b,c\} $
			\item[(c)] $A=\{a, c\}$ in $X=\{a, b, c\}$ with topology $\{X, \varnothing,\{a\},\{a, b\}\}$\\
			$ A' = \{b,c\} $
			\item[(d)] $A=\{b\}$ in $X=\{a, b, c\}$ with topology $\{X, \varnothing,\{a\},\{a, b\}\}$\\
			$ A' = \{a,c\} $
			\item[(e)] $ A=(-1,1) \cup\{2\}$ in the standard topology on $\mathbb{R}$\\
			$ A'=[-1,2] $
			\item[(f)] $A=(-1,1) \cup\{2\}$ in the lower limit topology on $\mathbb{R}$
			$ A'=[-1,2) $
			\item[(g)] $A=\left\{(x, 0) \in \mathbb{R}^{2} | x \in \mathbb{R}\right\}$ in $\mathbb{R}^{2}$ with the standard topology.\\
			$ A' = \R $
			\item[(h)] $A=\left\{(0, x) \in \mathbb{R}^{2} | x \in \mathbb{R}\right\}$ in $\mathbb{R}^{2}$ with the topology generated by the basis in Exercise 1.19\\
			
			\item[(i)] $\quad A=\left\{(x, 0) \in \mathbb{R}^{2} | x \in \mathbb{R}\right\}$ in $\mathbb{R}^{2}$ with the topology generated by the basis Exercise $1,19$\\
			$ A' = \R $
		\end{enumerate}
		\item[2.20] Prove Theorem 2.11 : Let $A$ be a subset of $R^{n}$ in the standard topology. If $x$ is a limit point of $A,$ then there is a sequence of points in $A$ that converges to $x$.
		\begin{proof}
			\wts{$ x \in A' \Rightarrow \exists (x_n)$}
			Let $ x\in A $ be a limit point and let $ n\in \Z_+ $. Then, for each $ x_n \in B(x,\frac{1}{n})\cap A -{x} $. Notice, $ x_n \not = x $ and $ x_n \in A $. Since, $ d(x_n,x)\leq \frac{1}{n} $, we have $ x_n $ converges to $ x $.\\
			Therefore,  If $x$ is a limit point of $A,$ then there is a sequence of points in $A$ that converges to $x$.
		\end{proof}
		\item[2.21] Determine the set of limit points of the set
		$$
		S=\left\{\left(x, \sin \left(\frac{1}{x}\right)\right) \in \mathbb{R}^{2} | 0<x \leq 1\right\}
		$$
		as a subset of $\mathbb{R}^{2}$ in the standard topology. (The closure of $S$ in the plane is known as the topologist's sine curve.)\\
		\begin{proof}
				Let $ y\in[-1,1] $ and $ p  = (0,y)$. Notice, for every neighborhood of radius $ r $, $ B(p,r) - \{p\}$ contain points in $ S $. Let $ n\in \R $ such that $ \frac{1}{2\pi n} < r $. 
				Then, $ sin(\frac{1}{x})  $ maps to all values of $ [-1,1] $, for some $ x\in (\frac{1}{2\pi(n+1)},\frac{1}{2\pi n}) $.\\ Thus, every point in $ S $ is a limit point
		\end{proof}
		
	\end{enumerate}
\end{document}