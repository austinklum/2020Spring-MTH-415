%This is a Latex file.
\documentclass[12pt]{article}

\usepackage{amsmath, amssymb, amsthm, amscd, amsfonts, dsfont}
\usepackage{mathrsfs}
\usepackage{graphicx}
\usepackage{verbatim}
\usepackage{enumerate}
\usepackage{url,hyperref}
\usepackage{comment}
\usepackage{multicol}
\usepackage{latexsym,fancyhdr}
\usepackage[margin=1in]{geometry}
\usepackage{lastpage} % Required to determine the last page for the footer
\usepackage{bbold}

\parindent 0pt

\pagestyle{fancy} \lhead{\sf MTH 415} \chead{\sf Defintions and Theorems}
\rhead{\sf Austin Klum} \lfoot{} \cfoot{} \rfoot{}

\newcommand{\C}{\mathbb{C}}
\newcommand{\R}{\mathbb{R}}
\newcommand{\Q}{\mathbb{Q}}
\newcommand{\Z}{\mathbb{Z}}
\newcommand{\N}{\mathbb{N}}

\renewcommand{\AA}{\mathcal{A}}
\newcommand{\BB}{\mathcal{B}}
\newcommand{\CC}{\mathcal{C}}
\newcommand{\DD}{\mathcal{D}}
\newcommand{\RR}{\mathcal{R}}
\renewcommand{\SS}{\mathcal{S}}
\newcommand{\TT}{\mathcal{T}}

\newcommand{\wts}[1]{\textit{\textcolor{blue}{WTS: #1}}\\}
\newcommand{\df}[1]{\subsection*{Def #1}}
\newcommand{\thm}[1]{\subsection*{Thm #1}}
\newcommand{\lem}[1]{\subsection*{Lem #1}}

\begin{document}
	\section*{Section 3.1}
	\df{3.1: Subspace Topology}
	DEFINITION 3.1. Let X be a topological space and let Y be a subset of $X .$\\
	Define $\mathcal{T}_{Y}=\{U \cap Y | U \text { is open in } X\} .$ This is called the subspace topology on $Y$ and, with this topology, $Y$ is called a subspace of $X .$ We say that $V \subset Y$ is open in $Y$ if $V$ is an open set in the subspace topology on $Y$.
	\df{3.2: Standard Topology on a subspace $Y$}
	DEFINITION 3.2. Let Y be a subset of $\mathbb{R}^{n}$. The standard topology on $Y$ is the topology that $Y$ inherits as a subspace of $\mathbb{R}^{n}$ with the standard topology.
	\df{3.3: Closed in a Subspace}
	DEFINITION 3.3. Let X be a topological space, and let $Y \subset X$ have the subspace topology. We say that a set $C \subset Y$ is closed in $Y$ if $C$ is closed in the subspace topology on $Y$
	\thm{3.4: Showing a set is closed in a subspace}
	THEOREM 3.4. Let X be a topological space, and let $Y \subset X$ have the subspace topology. Then $C \subset Y$ is closed in $Y$ if and only if $C=D \cap Y$ for some closed set $D$ in $X$
	\thm{3.5: Basis for subspace}
	THEOREM 3.5. Let X be a topological space and B be a basis for the topology on $X .$ If $Y \subset X,$ then the collection
	$$
	B_{Y}=\{B \cap Y | B \in \mathcal{B}\}
	$$
	is a basis for the subspace topology on $Y$
	\section*{Section 3.2}
	\df{3.6: Product Topology}
	DEFINITION 3.6. Let $X$ and $Y$ be topological spaces and $X \times Y$ be their product. The product topology on $X \times Y$ is the topology generated by the
	basis
	$$
	\mathcal{B}=\{U \times V | U \text { is open in } X \text { and } V \text { is open in } Y\}
	$$
	\thm{3.7: Basis for product of two topologies}
	THEOREM 3.7.  Let $X$ and $Y$ be topological spaces and $X \times Y$ be their product. Define $$
	\mathcal{B}:=\{U \times V | U \text { is open in } X \text { and } V \text { is open in } Y\}
	$$
	The collection $ \BB $ is a basis for a topology on $ X\times Y $
	\thm{3.8: Products of bases form a basis}
	THEOREM 3.8. If $\mathcal{C}$ is a basis for $X$ and $\mathcal{D}$ is a basis for $Y$, then
	$$
	\mathcal{E}=\{C \times D | C \in \mathcal{C} \text { and } D \in \mathcal{D}\}
	$$
	is a basis that generates the product topology on $X \times Y$.
	\thm{3.9: Subspace of a Product Topology}
	THEOREM 3.9. Let $X$ and $Y$ be topological spaces, and assume that $A \subset X$ and $B \subset Y .$ Then the topology on $A \times B$ as a subspace of the product $X \times Y$ is the same as the product topology on $A \times B,$ where A has the subspace topology inherited from $X,$ and $B$ has the subspace topology inherited from $Y .$
	\thm{3.10: Interior of Product Topologies}
	THEOREM 3.10. Let $ A  $ and $ B $ be subsets of topological spaces $X$ and $Y$, respectively. Then $\operatorname{Int}(A \times B)=\operatorname{Int}(A) \times \operatorname{Int}(B)$
	\section*{Section 3.3}
	\df{3.11: Quotient Topology, Quotient Map, and Quotient Space}
	DEFINITION 3.11. Let $ X $ be a topological space and $ A $ be a set (that is not necessarily a subset of $X$ ). Let $p: X \rightarrow A$ be a surjective map. Define a subset $ U $ of $ A $ to be open in $ A $ if and only if $p^{-1}(U)$ is open in $X .$ The resultant collection of open sets in $A$ is called the \textbf{quotient topology induced by $p,$ } and the function $ p $ is called a \textbf{quotient map}. The topological space $ A $ is called a \textbf{quotient space.}
	\thm{3.12: Quotient Maps induce a Quotient Topology}
	THEOREM 3.12. Let $p: X \rightarrow A$ be a quotient map. The quotient topology on $A$ induced by $p$ is a topology.
\end{document}